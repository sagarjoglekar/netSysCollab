\section{Abstract}
Emotions are a very fundamental part of how humans communicate. A major context of what we say is communicated using our body language and our facial expressions. Our paper tries to benchmark several expression recognition systems in the wild, using comparative studies on datasets and sample data. We also experiment the problem of emotion recognition using asymmetric convolutional network designs and show that mutually complimentary networks can be trained to improve the overall recognition performance of the problem. We further try to understand analytically what can make an ideal emotion recognition system and propose a combined approach. `