\section{ A Social approach to Sentiments: }
Sentiments are fundamental part of our day to day social interactions. A face to face social interaction is generally augmented with facial expression, body language and linguistic sentiment to convey the exact meta information. These properties are very human in nature and are mimicked in the social networks as well. Studies like \cite{Joo2014b} have explored the world of linguistic sentiment in social networks, by comparing several popular textual sentiment analysis methods used for analysing tweets. Our paper tries to explore a similar exercise for perceptual sentiment in social media. 
\par
When it comes to perceptual sentiments, there are two broad categories that could be explored. The first category looks at the perceptual sentiment evoked by a social media content. The second category talks about the actual latent perceptual sentiment that comes with the context of the content itself. We will discuss about the research problems about both these categories.

\subsection{ Evoked perceptual sentiment }
Several works have done in depth studies using methods like crowdsourcing to understand the different shades of a particular evoked emotion. Works like UrbanGems \cite{urbanGems} and StreetScore \cite{nikhil} use crowdsourcing methods to understand degrees of human sentiment evoked because of pictures of real urban neighbourhoods. Sentiments like the feeling of safety and aesthetics are especially hard to quantify and crowdsourcing helps the authors to do some interesting modelling. On the other hand there are papers like \cite{jeni20123d} by L. Jeni et.al. describe utility of actual facial expression detection for understanding content consumer reaction. Such approaches help us understand the very effect of a particular content on the consumer. 

\subsection{ Latent perceptual sentiment }
This approach is what this paper stresses on. By latent perception, we mean the hidden parameters, which are part of the very content. Social networks like reddit have specific sub-reddits that work on appealing to these types media sentiments that evoke emotions like empathy, disgust, contempt and love . One such popular sub-reddit is labelled R/aww which contains images and GIFs that showcase cute animals and animal behaviours. Another one called R/cringe appeals to the sentiments of awkwardness and discomfort by exhibiting videos and Gifs about people in awkward situations. These specific social channels are popular because the content shared over these channels have a certain type of latent sentimental response, which the consumers of these channels resonate with.
\par
Our paper focuses on this part of the story, and tries to survey and benchmark certain state of the art methodologies out there. We also propose certain hybrid approaches, which show that we can attain much better performance if a heuristic approach to combine certain methods is taken. 