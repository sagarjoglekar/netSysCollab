\section{ A Social approach to Emotion: }
Emotions are fundamental part of our day to day social interactions. A face to face social interaction is generally augmented with facial expression, body language and linguistic sentiment to convey the exact meta information. These properties are very human in nature and are mimicked in the social world as well. Studies like \cite{Joo2014b} have explored the world of linguistic sentiment in social networks, by comparing several popular sentiment analysis methods used for twitter analysis. Our paper tries to explore a similar exercise for perceptual emotion in social media. 
\par
When it comes to perceptual emotions, there are two broad categories of emotions that could be explored. The first category looks at the perceptual emotion evoked by a social media content. The second category talks about the actual latent perceptual emotion that comes with the context of the content itself. We will discuss about the research problems about both these categories.

\subsection{ Evoked perceptual emotion }
Several works have done in depth studies using methods like crowdsourcing to understand the different shades of a particular evoked emotion. Works like UrbanGems \cite{urbanGems} and StreetScore \cite{nikhil} use crowdsourcing methods to understand degrees of human emotions evoked because of pictures of real urban neighbourhoods. Emotions like the feeling of safety and aesthetics are especially hard to quantify and crowdsourcing helps the authors to do some interesting modelling. On the other hand there are papers like \cite{jeni20123d} by L. Jeni et.al. describe utility of actual facial expression detection for understanding content consumer reaction. Such approaches help us understand the very effect of a particular content on the consumer. 

\subsection{ Latent perceptual emotion }
This approach is what this paper stresses on. By latent perception, we mean the hidden emotional parameters, which are part of the very content. Social networks like reddit have specific sub-reddits that work on appeling to these types media content that evoke emotions like empathy, love . One such popular sub-reddit is labelled R/aww which contains images and GIFs that showcase cute animals and animal behaviours. These specific social channels are popular because the content shared over these channels have a certain type of latent emotional response. 
\par
Our paper focuses on this part of the story, and tries to survey and benchmark certain state of the art methodologies out there. We also propose certain approaches, which works as a hybrid and show that we can attain much better performance if a heuristic approach to combine certain methods is taken. 