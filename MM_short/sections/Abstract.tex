\section{Abstract}
In the past few years we witnessed the rise of the \textit{Selfie} phenomena. The world of online social networks was taken by a storm. There were several social, psychological and social computing studies trying to understand this phenomena. 
%In late 2012, a company was found on the basis of adding a new dimension to the concept of social media called Vine. Vine was solely based on the premise of sharing a short high impact video that delivers a message. Soon the service became popular and the bigger and more popular services like Instagram and Twitter started following their footsteps. They have now enabled sharing of limited duration video clips. 
Our work defines and validates a new phenomenon of Video-selfie which behaves much more like a still selfie but in the video realm. The paper tries to measure video-selfies and understand the Face to frame ratios. The ask certain questions about the ontological description of video selfies and correlation of those with video popularity. Finally the paper proposes a new framework to understand video selfies using \textit{Machine Intelligence} and validate it by benchmarking the predictors that arise from it. We do all our experiments on a crawled dataset gathered from the popular video service Vine, and then we do comparative analysis of a popular Instagram still selfie dataset. 
