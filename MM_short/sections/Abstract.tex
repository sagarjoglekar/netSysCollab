\section{Abstract}
In the past few years we witnessed the \textit Selfie phenomena. The world of online social networks was taken by a storm. There were several social, psychological and social computing studies trying to understand this phenomena. In late 2012, a company was found on the basis of adding a new dimension to the concept of social media. Vine was solely based on the premise of sharing a short high impact video that delivers a message. Soon the service became popular and the more popular services like Instagram and Twitter have started following the footsteps of this service and have enabled sharing of limited duration video clips. Our paper tries to measure this phenomenon and makes certain studies about how a particular vine-like video gains popularity. We further look at this through the lense of affective computing and propose a new framework to understand human affects in the budding research field of social \textit Artificial Intelligence
