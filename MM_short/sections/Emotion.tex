\section{Sentiments in Social Networks}
Sentiments are fundamental part of our day to day social interactions. A face to face social interaction is generally augmented with facial expression, body language and linguistic sentiment to convey the exact meta information. These properties are very human in nature and are mimicked in the social networks as well. Studies like \cite{Joo2014b} have explored the world of linguistic sentiment in social networks, by comparing several popular textual sentiment analysis methods used for analysing tweets. 
\par
When it comes to social media shared over OSNs, the analysis becomes complicated. This is because social media involves higher dimensional messages like Videos, audio and Images. 
%There are problems in the fields of Computer Vision and Pattern recognition which strive at Content based Image Retrieval (CIBR) \cite{CBIRSurvey} which aim at understanding the structure of images and delve deeper into what makes an image pleasing or appealing. The methods often take the route of designing low level image features based on saptio-temporal characteristics as well as colour, shape and structure of the image components. They generally end up designing a classifier which comes up with a higher level representation of the image as a function of low level features. However these methods are over engineered and highly unsuitable when it comes to social media because of the very nature of the media. 
Moreover the media shared has a very human centric content. That means the media will involve a lot of faces, poses and affective means of communications. The studies done in \cite{Souza2015} show that there has been 900 times increase in the number of selfies over Instagram in just 2 years. Another recent paper \cite{goodSelfie} states that everyday more than 90 million selfies are taken using just the Android clients out there and are uploaded on Instagram. We collected Vine social network data, which is a popular social network that uses short 6 seconds videos as a medium. In that dataset we found that  one in ever three video in the popular videos category contain human faces for more than 60 percent of the frame length. 
The very human centric nature of the media shared over these networks, make sentiments and human affects an integral part. These mediums 
%However sentiments are human quantities perceived or experienced by the human consumer of the content. And these perceptions are now manifesting in the social world as a medium for information dissemination or engagement or plane interactions. 
When it comes to perceptual sentiments, there are two broad categories that could be explored. The first category looks at the perceptual sentiment evoked by a social media content. The second category talks about the actual latent perceptual sentiment that comes with the context of the content itself. We will discuss about the research problems about both these categories.

\subsection{ Evoked perceptual sentiment }
Several works have done in depth studies using methods like crowdsourcing to understand the different shades of a particular evoked emotion. Works like UrbanGems \cite{urbanGems} and StreetScore \cite{nikhil} use crowdsourcing methods to understand degrees of human sentiment evoked because of pictures of real urban neighbourhoods. Sentiments like the feeling of safety and aesthetics are especially hard to quantify and crowdsourcing helps the authors to do some interesting modelling. On the other hand there are papers like \cite{jeni20123d} by L. Jeni et.al. describe utility of actual facial expression detection for understanding content consumer reaction. Such approaches help us understand the very effect of a particular content on the consumer. 

\subsection{ Latent perceptual sentiment }
This approach is what this paper stresses on. By latent perception, we mean the hidden parameters, which are part of the very content. Social networks like reddit have specific sub-reddits that work on appealing to these types media sentiments that evoke emotions like empathy, disgust, contempt and love . One such popular sub-reddit is labelled R/aww which contains images and GIFs that showcase cute animals and animal behaviours. Another one called R/cringe appeals to the sentiments of awkwardness and discomfort by exhibiting videos and Gifs about people in awkward situations. These specific social channels are popular because the content shared over these channels have a certain type of latent sentimental response, which the consumers of these channels resonate with.
\par
Our paper focuses on this part of the story, and tries to survey and benchmark certain state of the art methodologies out there. We also propose certain hybrid approaches, which show that we can attain much better performance if a heuristic approach to combine certain methods is taken. 