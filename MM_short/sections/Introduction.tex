\section{Introduction}
Online social networks (OSNs) have seen a massive surge in usage over the past decade. The surge is going hand in hand with the explosion of smart phone industry. More and more social interactions are now driven by media contents like selfies, group selfies and selfie videos because of the ubiquitous nature of cameras. A sharp changes in cultural aspects of online social interactions are evident and have also been studied in detail in papers like \cite{Souza2015}. Naturally, properties of real life interactions are now going to emerge in virtual social interactions. A study by Daniel Perez et.al \cite{7175072} observes that non-verbal communication has a strong effect on how a person is perceived at times of real world situations like interviews. These real world behaviours can also be extended to the virtual social world. With the rise of social media networks like Vine and Instagram, human to human non-verbal interactions have another dimension to manifest. 
The \cite{Souza2015} paper explores several of these properties amongst Instagram users, where they explore correlation of facial orientation, poses and smiles with parameters like country of origin of the selfie user, post frequency, likes received , number of faces in the pictures, gender and smile scores. Such studies give us interesting insights about the sharp rising OSN phenomena of selfies. The study also states that more than 50 percent of photos shared on Instagram, fall under the category if selfies. These numbers are pretty much consistent amongst video OSNs like Vine, where in the protagonist of the video is in focus for most of the 7 seconds of the vine. 
\par
Taking into consideration these measurements from previous works, there seems an urgent need to bring these mediums under the overarching problem of sentiment analysis in social networks. That would allow the social network community to analyse and measure the dynamics of the world of social media in a different dimension