\section{Introduction}
Online social networks (OSNs) have seen a massive surge in usage over the past decade. The surge is going hand in hand with the explosion of smart phone industry. More and more social interactions are now driven by media contents like selfies, group selfies and videos because of the ubiquitous nature of cameras. A sharp change in cultural aspects of online social interactions are evident and have also been studied in detail in papers like \cite{Souza2015}. 
With the rise of social media networks like Vine and Instagram, human to human non-verbal interactions have another dimension to manifest. One of the predominant modality of self expression arose from this boom in social media, and that was the Selfie. The \cite{Souza2015} paper explores several of the properties of selfie amongst Instagram users, where they explore correlation of facial orientation, poses and smiles with parameters like country of origin of the selfie user, post frequency, likes received , number of faces in the pictures, gender and smile scores. Such studies give us interesting insights about the sharp rising OSN phenomena of selfies. The study also states that more than 50 percent of photos shared on Instagram, fall under the category if selfies. 
\par
A major change in these behaviours was seen when the social network called Vine was launched in 2012. Vine adds another dimension to the act of self expression, where the users can record a 7 second long video and post it online and get engagement from peers. This service got so popular that the larger services like Instagram and Twitter also added the feature of short videos to their services. These developments just indicate that the fact that videos add more dimensions to the act of self expressions, they are being used heavily by the millennials and the young adult generation. Hence our paper asks the following questions regarding this medium of expressions

\begin{enumerate}
\item Video Selfie: Is there something analogous to a still selfie in this new medium? How can we define it?
\item The Influence: Do followers react differently to video selfies compared to non-selfies ? 
\item Does the face says it all: Do human affects influence the popularity of a video selfie? 
\item What do the vines say: What are the most common ontological descriptions that describe vines?
\end{enumerate}

To our best of knowledge, this would be a first of its kind investigation into this type of medium of expression. More over our paper tries to employ techniques like adjective noun pairs \cite{SentiBank} and convolutional neural networks \cite{Campos:2015:DDS:2813524.2813530} to understand human affects in video. We work on real data collected from a popular social media service called Vine over one month. In the following sections we will try to describe our approach
