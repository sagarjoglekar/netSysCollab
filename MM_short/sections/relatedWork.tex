\section{ Related Work }
There have been several attempts to understand the phenomena of self expression. With the rise of selfie, the expression exposed several facets of human nature. \cite{Souza2015} looked at the phenomenon of selfie as a whole. There the authors explain social structures , temporal dynamics, demographics and memes using Instagram datasets of selfies. There were other works in this area \cite{hu2014we} which explore the content itself. They talk about what kind of content is posted by specific categories of users. Also they try to understand how different types if content relate to the number and types of followers an account gathers. Another interesting work \cite{goodSelfie} goes in a different direction and tries to understand whys and whats of a perfect selfie. They try to analyse the ontological aspects of Instagram selfies using Sentibank \cite{SentiBank} and discuss the salient characteristics of popular selfies. They also employ Deeplearning model of Imagenet \cite{NIPS2012_4824} which can detect 1000 classes of objects in an image. These approaches help in understanding what are the most common themes of still selfies on Instagram.  These works ask a bigger question in the realm of social media analysis. What aspects of media appeal to humans. What makes a media more or less consumable.
\par
There were social media analysis of media in social network drawing conclusion on certain types of media gaining more popularity and engagement. The work by Bakshi et.al \cite{Bakhshi:2014:FEU:2611105.2557403} shows that a study of instagram conclusively shows that photos with faces attract more likes and comments. We try to explore this hypothesis in videos. But first we would like to introduce some of our dataset numbers
