\section{ Related Work }
There have been several attempts to understand the phenomena of self expression. With the rise of selfie, the expression exposed several facets of human nature. \cite{Souza2015} looked at the phenomenon of selfie as a whole. There the authors explain social structures , temporal dynamics, demographics and memes using Instagram datasets of selfies. There were other works in this area \cite{hu2014we} which explore the content itself. They talk about what kind of content is posted by specific categories of users. Also they try to understand how different types if content relate to the number and types of followers an account gathers. Another interesting work \cite{goodSelfie} goes in a different direction and tries to understand whys and whats of a perfect selfie. They try to analyse adjective noun pairs using Sentibank \cite{SentiBank} and discuss the salient characteristics of popular selfies. These works ask a bigger question in the realm of social media analysis. What aspects of media appeal to humans. What makes a media more or less consumable. We look at this problem from an angle of machine intelligence question. Can you learn the higher level representation that makes a video or an image disseminate faster in the network. To begin our work, we try analysing Vine social network and the content uploaded over it. The following section would delve deeper into our work.