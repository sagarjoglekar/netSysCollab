\section{Abstract}
\begin{comment}
The Art of story telling could be attributed to be one of the most ancient arts. One might give a considerable chunk of credit for the possibility of humans to trace their footsteps across history, to this very art. This paper tries to explore the art of story telling in the realm of Online Social Networks (OSNs) and online social media. Our work hypotheses that the presence of the well known screen play graph, which directs the over all sentiment of a movie or a drama through time, is very well present in the micro videos posted on the newly available mediums of Vine, Instragram and twitter. The In the due course of the work done for the paper, we crawled a popular social media network called Vine for almost 2 months and collected over 12000 unique vine videos and their meta data. We try to take an approach based on perceptual sentiment in social media and hypothesize existence of story lines in perceptual sentiments. We use deep learning tools to detect sentiment values of videos frames and eventually show to a reasonable extent, that perceptual sentiments do follow popular screen-writing theories. The sentiment transitions across these short but high impact videos, do follow certain trends which could be explained from popular screenplay writing theories. The paper also evaluates correlations of individual perceptual sentiments of videos with popularity metrics of the videos. The paper validates presence of genres based on perceptual sentiments of videos and tries to explain them using some popular screenplay techniques.
\end{comment}
The human need to tell, listen and experience stories in various forms has almost driven various forms of art for thousands of years. From different dramatic dance forms, to puppets, to the modalities enabled by the new age i.e. social media; the basic driver has always been an interesting story. In this paper we explore this very aspect of social media, by trying to propose and validate theories around how videos shared over social media follow popular screenplay patterns and thrive in the digital realm. In the due course of analysis, the paper shows that the most ristricted of videos, also follow story lines and that there are computational ways to classify them. We use human percieved sentiments, and analyse the content in sentiment space. We propose and validate our observations using over 12 thousand videos crawled from popular social media services like Vine and Youtube.
 