\section{Abstract}
This paper tries to explore the art of story telling in the realm of Online Social Networks (OSNs) and online social media. Our work hypotheises that the presence of the well known screen play graph, which directs the over all sentiment of a movie or a drama through time, is very well present in the micro videos posted on the newly available mediums of Vine, instragram and twitter. The In the due course of the work done for the paper, we crawled a popular social media network called Vine for almost 2 months and collected over 12000 unique vine videos and their meta data. We try to take an approach based on perceptual sentiment in social media and hypothesize existence of story lines in perceptual sentiments. We use deep learning tools to detect sentiment values of videos frames and eventually show to a reasonable extent, that perceptual sentiments do follow popular screenwriting theories. The sentiment transitions across these short but high impact videos, do follow certain trends which could be explained from popular screenplay writing theories. The paper also evaluates correlations of individual perceptual sentiments of videos with popularity metrics of the videos. The paper validates presence of generes based on perceptual sentiments of videos and tries to explain them using some popular screenplay techniques.