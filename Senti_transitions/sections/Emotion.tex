\section{Sentiments in Media}
Sentiments are fundamental part of our day to day social interactions. A face to face social interaction is generally augmented with facial expression, body language and linguistic sentiment to convey the exact meta information. These properties are very human in nature and are mimicked in the social networks as well. Studies like \cite{Joo2014b} have explored the world of linguistic sentiment in social networks, by comparing several popular textual sentiment analysis methods used for analysing tweets. 
\par
When it comes to social media, the analysis becomes complicated. This is because social media involves higher dimensional messages like Videos, audio and Images. Moreover the media shared has a very human centric content. That means the media will involve a lot of faces, poses and affective means of communications. The studies done in \cite{Souza2015} show that there has been 900 times increase in the number of selfies over Instagram in just 2 years. Another recent paper \cite{goodSelfie} states that everyday more than 90 million selfies are taken using just the Android clients out there and are uploaded on Instagram. We collected Vine social network data, which is a popular social network that uses short 6 seconds videos as a medium. In that dataset we found that  one in ever three video in the popular videos category contain human faces for more than 60 percent of the frame length. 
The very human centric nature of the media shared over these networks, make sentiments and human affects an integral part. These mediums 
When it comes to perceptual sentiments, there are two broad categories that could be explored. The first category looks at the perceptual sentiment evoked by a social media content. The second category talks about the actual latent perceptual sentiment that comes with the context of the content itself. We will discuss about the research problems about both these categories.

