\section{Introduction}
The Art of story telling could be attributed to be one of the most ancient arts. One might give a considerable chunk of credit for the possibility of humans to trace their footsteps across history, to this very art. It could be in the form of neolithic paintings to the egyptian hieroglyphs or among the elaborate epics of Illiad and oddyssey to the elaborate power plays of the Game of thrones, humans have always strived to record or create elaborate plots and stories. The human need of transferring experiences to others in different forms of creative arts, has ever so created the world as interesting as we see it. 
\par
The art of story telling is one such creative art, which has spawned and transformed several industries, including the very important entertainment industry. With the invention of the internet and the online social networks, entertainment industry has gone through several rejuvination cycles. The latest of these cycles was powered by the proliferation of services like Youtube, Netflix, Hulu and others like them. Our work in this paper attempts to explore the validity of screenplay and screen writing theories in online social media. More specifically we try to measure perceptual sentiment across a video shared over an OSN, and look for strong evidence in frame sentiments to support generic screen writing theories. We eploy some of the most promising ideas of Visual and perceptual sentiment measurements \cite{SentiBank} and couple them with deep learning frameworks for increased generalized performance. Through this work we were able to detect strong clusters of trends of visually percieved sentiments that convey a strong overall story, independently from the actual audio track content. 
