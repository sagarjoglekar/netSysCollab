\section{Screenplay Theories:}
Every art, despite being a subjective field, does have certain theories that have emerged over time, which are taught to the novices of the field. These theories propose a boad structure or an approach towards creating the art. Screenwriting is not an exception to this rule. Over decades of excellent cinema and drama, writers and directors have come up with a genral map of how a story is told an how the overall sentiment of a story fluctuates up untill the climax. Let us briefly discuss some of the described categories. 

\subsection{Aristotle's Three Act Structure}{
This structure of screenplay is probably the most well studied and has been in the literature since its first mention by Aristotle himself. The structure of the screenplay is generally divided into 3 sections or acts where there is a change in sentiment of the whole story at during every act. This happens generally towards the end of first act because of a situation that poses a challenge or risk to the protagonist of the story. The tension and the protagonist's struggles build up during the second act. Finally the climax of the screenplay which begins towards the end of second act or the beginning of the third, creates a closure for the protagonist by either release of tension or through resolution of a challenge which may come in form of a villian. Throughout the acts for this sort of a structure, the sentiment of the story transitions drastically, going from positive to negative to positive again. Such a transition keeps the viewer engaged and engrossed. }

\subsection{Hero's journey}{
Such a screenplay structure is ominipresent across classic stories and typical rags-to-riches stories. The protagonist of the screenplay generally encounters a call to adventure or a proposition of adversity, which he generally accepts and goes on a fluctuationary trajectory of sentiments. The protagonist generally ends up a hero by the climax and the story generally ends on a high sentiment.}

\subsection{Syd Field's paradigm}{
Syd Field in his book Screenplay introduced his new concept of paradigms. This is the first theory that explains screenplay writing as a series of plot points and is the most used structure for screenplay out in the modern world. A plot point in a screenplay is a situation in the story that changes the narrative of the story and inevitably the sentiment of the story in some way. }

\par
The main contribution of this work, is to suggest that these popular screenplay theories are present in the new age of entertainment videos, which are the micro-videos , found on services like Vine, Instagram and Twitter. Doing so opens up a new paradigm of understanding the field of computational screenplay classification.