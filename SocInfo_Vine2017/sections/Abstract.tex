\begin{abstract}
%Online content creators, both professional ones and amateur authors of user-generated content, need to answer an important question: How does one capture viewers' attention in this age of information overload? 

%Several content-driven platforms have adopted the `microvideo' format, a new form of short video that is constrained in duration, typically at most 5-10 seconds long. Microvideos are typically viewed through mobile apps, and are presented to viewers as a long list of videos that can be scrolled through. How should microvideo creators  capture viewers' attention in the short attention span? Does quality of content matter? Or is there greater support for  ``rich gets richer'' theories, which suggest a self-perpetuating phenomenon where content from users with large numbers of followers stands a greater chance of becoming popular?  To the extent that quality matters, what aspect of the video -- aesthetics or affect --  is critical to ensuring popularity? 

Several content-driven platforms have adopted the `micro video' format, a new form of short video that is constrained in duration, typically at most 5-10 seconds long. Micro videos are typically viewed through mobile apps, and are presented to viewers as a long list of videos that can be scrolled through. How should micro video creators  capture viewers' attention in the short attention span? Does quality of content matter? Or do social effects predominate, giving content from users with large numbers of followers a greater chance of becoming popular?  To the extent that quality matters, what aspect of the video -- aesthetics or affect --  is critical to ensuring user engagement? 


We examine these questions using a snapshot of nearly all ($>120,000$) videos  uploaded to globally accessible channels on the micro video platform Vine over an 8 week period. We find that although social factors do affect engagement, content quality becomes equally important at the top end of the engagement scale. Furthermore, using the temporal aspects of video, we verify that decisions are made quickly, and that first impressions matter more, with the first seconds of the video  typically being of higher quality and having a large effect on overall user engagement. We verify these data-driven insights with a user study from 115 respondents, confirming that users tend to engage with micro videos based on ``first sight'', and that users see this format as a more immediate and less professional medium than traditional user-generated video (e.g., YouTube) or user-generated images (e.g., Flickr). %We demonstrate that despite being a video format,  micro videos on Vine fall on a spectrum in between the more familiar user-generated images on sites like Flickr, and user-generated videos on sites like YouTube. 


%In this article we examine the relevence of several aesthetic, sentimental and social features with popularity of microvideos. We look at influence of social and aesthetic features on how a vine video performs in the social world. We also look at the affective component of the videos to see if there are any peculiar sentiments related to the success of a vine video 
\end{abstract}