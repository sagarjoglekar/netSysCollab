Overall, our results might confirm the theory that Vine is a novel form of expression that is different from both images and videos~\cite{pogue13}. To further test this conjecture, we perform a last experiment that compares Vine with other media platforms using a computational approach.

More specifically, we compare the 18 computational aesthetics features from \S\ref{sec:features) of POP12K vines with the corresponding features from a sample of 1000 popular\footnote{i.e., with high  ``interestingness'' rating -- \scriptsize https://www.flickr.com/explore/interesting rating}  images from the MIR-Flickr dataset~\cite{huiskes08}, and 419 viral YouTube videos~\cite{Jiang:2014:VVS:2578726.2578754}. As expected, we find that Vine videos show different aesthetic behaviour from both Flickr and YouTube, although they are more stylistically similar to YouTube videos than Flickr images.

We then examine platform differences in terms of the visual objects they focus on.  We use deep learning-based object detectors from ImageNet~\cite{krizhevsky2012imagenet} and look at the frequency of visual objects in each platform (e.g. how often does a cat appear in a YouTube video?). 
Vine can be clearly distinguished from the other 2 media due to the higher presence of objects related to celebrations, fun and entertainment (\emph{academic dress}, \emph{wig}, \emph{tv}, \emph{sunglasses}). Viral Youtube videos prefer popular subjects such as kids (\emph{diaper}), cars, and  violent scenes (\emph{punching}, \emph{neck brace}). Finally, Flickr pictures can be distinguished with the presence of visual concepts typical of photographic sceneries (\emph{lakeside}, \emph{seashore}). 

These results could be taken to indicate that the three platforms are used for different purposes, with micro-videos being more of an immediate medium, `capturing the moment' as it happens. This may help explain the nature of user engagement we observed throughout of the paper.

