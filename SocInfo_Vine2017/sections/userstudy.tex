\section{User study}
\label{sec:userstudy}
To complement the data analysis and gain deeper insight into what drives user actions and engagement, we designed an anonymous user study
which captures user behavior when engaging with micro-videos. 

\subsection{Survey methodology}
We initially recruited undergraduate students, obtaining about 33 responses.
Subsequently, we tweeted the survey out to the Official Vine Twitter account and to the accounts of Vine and Instagram users, in order to gain further exposure amongst 
users of these platforms. In all, 115 users responded to our survey. Table \ref{tbl:survey} summarizes the respondents' demography and usage preferences. Most questions asked were to be answered either on a 5-point Likert scale  (Strongly agree to Strongly Disagree), and or in a semantic differential format, with three options to choose from. 
%\footnote{Survey can be found here :  https://goo.gl/forms/lHEea9EzYCyRy3jQ2}.

\begin{table}[hbt]
	\centering
	\begin{tabular}{l|c}
		\thead{Attribute} & \thead{\shortstack{Value}} \\
		\hline
		Male respondents (\%) & 44.2 \\
		Female respondents (\%) & 55.8  \\
		\hline
		Age Demography (\%) &\\
		\hline
		 18-24 & 43.4 \\
		 25-31 & 34.5 \\
		 32-40 & 14.2 \\
		 40+    & 8 \\
		\hline
%		Platforms used (\%) &\\
%		\hline
%		Instagram & 67.3 \\
%		Snapchat stories & 27.4 \\
%		%Whatsapp stories & 97 \\
%		Vine & 2 \\
%		\hline
	\end{tabular}
	\caption{Summary of survey responses}
	\label{tbl:survey}
\end{table}


%We asked users to select the micro-video platforms which the users are most exposed to. Snapchat and Instagram polled highest in the sample population.
%Although Vine polled  low in terms of declared usage, the reason being that Vine was declared to be discontinued as of the date when the survey was conducted\footnote{Moreover our survey was primarily answered by respondents from Europe and Asia, where vine had very limited penetration.}, almost 70\% of users declared to have used Instagram %which also hosts 
%micro videos. Hence this survey can be considered to be a good proxy for understanding ground truth user engagement with micro-videos. 

\subsection{Validation of data-driven results}
To understand engagement with micro-videos and validate the findings that emerged from our analysis of \textbf{RQ1} and \textbf{RQ2}, the survey asked the following 5 questions
to be answered on a 5 level Likert scale (strongly agree - strongly disagree):
\begin{enumerate}
\small 
\item [A] I tend to like/comment on videos from friends rather than from strangers
\item [B] I always form an opinion of a video in the initial few seconds, once the video starts playing
\item [C] I rarely watch short videos (Snapchats, stories) , all the way to the end.
%\item [D] I don't really care about the quality of the micro-video or stories, as long as I like the content
\item [D] I prefer to watch short videos of humans on these platforms. E.g.\ I like to see a person talking/expressing rather than outdoor scenery, or Cats.
\end{enumerate}

Almost 66\% users agreed to question \textbf{A}, which reaffirms the tendency of socially embedded users being able to get high engagement scores. 44\% of users agreed to question \textbf{B} (and further $\approx$ 30\% users were neutral) and 38\% users agreed with statement \textbf{C}, supporting the observed the ``primacy of the first seconds'' effect.
%55\% users agreed to statement \textbf{D}, suggesting a possible reason why . 
Contrary to our findings regarding faces, only 34\% users agreed with statement  \textbf{D} (although a further 39\% remained neutral; thus only a minority 27\% of users disagreed or disagreed strongly). Such result might suggest that for many users, our attraction towards face shapes is innate \cite{slater1998innate} and people do not consciously engage more with faces. %While further study is required to confirm this conjecture, we find that on a sample of 6000 Instagram micro videos, the top 10 percent of videos by popularity had 

\subsection{Understanding what matters to users}
The next part of the survey went beyond confirming the data-driven analysis by asking the respondents how their behavior changed when it comes to \emph{acting} on a video they engaged with, i.e., when do they like/forward, comment or stop playing the video? 44\% like, comment or share videos only after finishing watching it, and but a sizable 56\% agreed that they do so in the middle of watching the video itself, or right at the beginning (19\% share at the beginning. 37\% somewhere in the middle); again pointing to the need for capturing users in the initial parts of the video. 

Interestingly, a majority of 55\% of respondents agreed (on a 5 point Likert scale) to the statement: ``I don't really care about the quality of the micro-video or stories, as long as I like the content''. This result, together with the previous answers seems to imply that   the fall off in content quality in the latter parts of micro videos does not negatively impact user engagement. However, users do see a difference between micro videos and ``traditional'' (and older) user generated content platforms such as YouTube: an overwhelming 75\% of respondents rate the production quality of YouTube videos quality as more professional than micro videos.

% \subsection{Where do micro videos fit in}
% Finally, several 
	