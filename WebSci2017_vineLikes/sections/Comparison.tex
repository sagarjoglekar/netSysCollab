
\section{Micro-videos as a new form of expression}
%To understand whether the constraint short duration format of videos, or micro-videos as we call it throughout the paper, is a ubiquitous form of expression, we do some higher level analysis of the Instagram dataset we crawled. 
%Vine is primarily a micro video platform, but is micro video a ubiquitous form of expression, and a prominent phenomenon across platforms? To address this quesiton, we analysed the videos in our Instagram dataset according to their duration. Interestingly, we found that 75\% of Instagram videos are shorter 20 seconds, and the median duration stands at 12.91 seconds. This suggests that, even with Instagram's relaxed length constraint, users prefer shorter-length videos. With this in mind, in this Section we explore the novelty of the micro-videos by comparing this form of expression with other traditional media forms using popular computational methods.





%\begin{figure}[htb]
%	\includegraphics[width=0.9\columnwidth]{plots/VideoDurationInsta}
%	\caption{A CDF of 15k videos crawled from Instagram. The CDF shows that more than 65\% of videos have duration less than 20 seconds }
%	\label{fig:VideoDurationInsta}
%\end{figure}
%To explore this question, we analyze the INSTA15K dataset, to understand the distribution of video length and engagement. 
%The results of the two dimensional distribution can be seen in Figure \ref{fig:InstaDistribution}. Content creators and consumers on Instagram seem to have converged on a duration length of videos shared on Instagram.  70\% videos created on Instagram fall between the duration range of 5 seconds to 15 seconds. This further supports the hypothesis of micro-videos being a new form of expression. This also motivates the study of a bigger and more diverse dataset pf micro-videos at our disposal.


\label{sec:new-vine-old-bottle}

\vspace{-2mm}
\begin{figure}[!htb]
\centering
\includegraphics[width=\columnwidth]{plots/comparison/table}
\caption{\textsl{ Aggregated symmetric KL Divergences between platforms over all features for different feature groups.}}
	\label{fig:comparisontable}
	\vspace{-2mm}

\end{figure}


%As pointed out  by previous work \cite{redi20146}, micro-videos have been often called a ``new form of expression'' that goes beyond the traditional video medium.  Vine constraints (short length, limited editing tools) offer an unprecedented number of possibilities for creative visual artists. Tech journalists \footnote{\small \url{https://www.scientificamerican.com/article/why-micro-movies-so-popular-today/}} have indeed theorised that micro-videos cannot be even categorised as videos: their goal is to capture a single moment, thus making them ideally close to the photographic medium, or  \emph{``neither photo nor video but something in between, with artistic merits all its own''}. 

To address \textbf{RQ1} and understand the position of Vine in the user generated content spectrum, %and the different features of its videos, 
we begin by comparing Vine with other media platforms, %image-based platform Flickr and video platform YouTube, 
and test the conjecture that Vine is a new form of expression that is different from both images and videos~\cite{pogue13}. For this preliminary study, we focus on platforms which have engaged large numbers of users, and compare our dataset of popular Vines with datasets of popular images from Flickr and videos on YouTube\footnote{Similar results were obtained replacing the popular videos with videos from the ALL120K dataset, but are not shown here.}.   

More specifically, we compare the features (See Table~\ref{tab:Features_table}) of POP12K vines with the corresponding features from a sample of 1000 popular\footnote{i.e., with high  ``interestingness'' rating -- \scriptsize https://www.flickr.com/explore/interesting rating}  images from the MIR-Flickr dataset~\cite{huiskes08}, and 419 viral YouTube videos~\cite{Jiang:2014:VVS:2578726.2578754}. To compare all the datasets as images, we sample the videos at the rate of 1 image per second producing several still images, whose parameters we study. 

We compare these datasets along three visual feature dimensions: visual quality, audio quality, and visual semantics.
Our methodology is straightforward: For a given platform and feature, we can compute the  marginal probability distribution of the feature across all videos (images) of the platform. For a given platform and feature, these distributions allow us to compute the probability of a random video (image) from that platform having a particular feature value. For a given feature, we can measure the differences between any pair of platforms by computing the symmetric Kullback-Liebler (K-L) divergence between the corresponding distributions of the feature. For a given pair of platforms, the average of the K-L divergence values across all features tells us how far apart the two platforms are in feature space. Fig.~\ref{fig:comparisontable} shows the differences between the platform across the visual, audio and object (or visual semantic) features described previously (Table~\ref{tab:Features_table}). We discuss this further by focusing on the features with the highest K-L divergence across each category of features: 

%We describe each Flickr image, YouTube video or Vine micro video with the 18 computational aesthetics features from \S\ref{sec:features} (Table~\ref{tab:Features_table}). Then, we compute the  marginal probability distribution of these features for each platform. %These distributions capture the platforms' dominant visual aesthetics patterns (e.g. what are the common brightness values in YouTube videos? Or the distribution of Colorfulness in Vine?). 



%We measure the distance of the marginal distributions of a feature across platforms  (e.g. Rule of Thirds in Flickr vs.\ rule of Thirds in Vine) using symmetric Kullback-Liebler (K-L) divergence. For a given pair of platforms, the average of the K-L divergence values across all features tells us how far apart the two platforms are in the aesthetics feature space.  %\mr{I think there was a commend in WWW clarity of this }

%To obtain a description of the visual aesthetics trends for each medium, we aggregate the computed features at a platform-level. To do so, for each dataset, we compute the feature marginal probability distribution. This reflects the platforms' dominant visual aesthetics patterns (e.g. what are the common brightness values in YouTube videos? Or the distribution of Colorfulness in Vine?). To compare distributions across platforms (e.g. Rule of Thirds in Flickr VS Rule of Thirds in Vine), we then use symmetric Kullback-Lieber (KL) divergence, which reflects the distance of 2 probability distributions. The average of such KL divergence values  tells us how far platforms are in the visual aesthetics feature space. We report the results of this analysis in Fig.~\ref{fig:comparisontable}: as expected, Vine videos show different visual aesthetic behaviour from both Flickr and Youtube, although more stylistically similar to long videos. 

\begin{figure}[!htb]
\centering
\includegraphics[width=\columnwidth]{plots/comparison/aesthetics}
\caption{\textsl{ Distributions of the most diverging aesthetic features across platforms.}}
\label{fig:comparison_aesthetics}
\end{figure}

\noindent\textbf{Visual quality comparison}
To describe visual quality, we describe each Flickr image, YouTube video or Vine micro video with the 18 computational aesthetics features from \S\ref{sec:features} (Table~\ref{tab:Features_table}). 
As expected, we find (Fig.~\ref{fig:comparisontable})  that Vine videos show different aesthetic behaviour from both Flickr and YouTube, although they are more stylistically similar to YouTube videos than Flickr images. Taking a closer look at the features with the highest KL divergence values across all platform pairs, we notice that, in practice, Flickr aesthetic features reflect the behaviour of a ``professional'' medium. At the other end of the spectrum, micro videos show patterns of less professional use, typical  of the  user-generated, mobile-first Vine content. Youtube lies in the middle. More specifically, we notice the following (See Fig. \ref{fig:comparison_aesthetics}). (1) \emph{Colorfulness:}  Flickr photos tend to have a higher color diversity, while Vine and Youtube tend to have less saturated colors. (2) \emph{Exposure:}  Flickr pictures tend to have a good balance between Left and Right  pixel intensities, typical of high quality images; unlike Flickr, Youtube and Vine videos show unbalanced exposure. (3) ) \emph{Rule of Thirds:}  Unlike Vine and Youtube, some Flickr pictures tend to deviate from the standard rule of thirds, as it occasionally happens for professional, artistic pictures~\cite{freeman2007photographer}.% On the other hand, the moving images of Vine and Youtube tend to stick to the Rule of Thirds heuristic. 
(4) \emph{Sharpness} is probably one of the most important properties of high quality visual content. Due to its mobile-based nature, Vine videos tend to have almost no sharp pixels, while the professional expertise of Flickr photographers is clearly exposed by the higher percentage of sharp pixels.

%%%%%%%%%%%%%%%%%%%% TEMPORARILY COMMENTED OUT  
%%\mr{this is raw from the insight sec, still need to do a pass on it, just tied it to the rest in the firs sentence}
%To further explore the relation between Vine and aesthetics, we compare the aesthetic features of frames sampled from both high engagement micro videos and low engagement videos, with highly aesthetic images (rated $>$ 6 on a scale from 0 to 7), taken from the dataset sourced from photo.net \cite{datta2008algorithmic}. %These images were rated for aesthetic appeal for the research involved and these ratings are also used while selecting the images for our comparison. We only choose images with median ratings of 6 \ns{Can you see if we get the same result for median ratings of a=4+? Choosing the lowest value of a makes the case stronger, and helps understand exactly how bad vine videos are} or above on aesthetic scale of 0 to 7.  
%Table~\ref{aesthetic_table} shows the comparison of means and medians of several of these aesthetic features compared to the highly aesthetic dataset. We can observe that popular and unpopular vines show similar values on most parameters considered, yet the scores are way below the values for most  aesthetically pleasing images from photo.net. %Thus, we conclude that whiles aesthetics in micro videos matter, we would not call them aesthetically pleasing. 
%
%\begin{table}
%\caption{List of Aesthetic parameters computed for highly rated aesthetic images, Popular videos and unpopular videos. Most parameters have no bias towards either popular or unpopular videos}
%
%\resizebox{\linewidth}{!}{%
%\begin{tabular}{|l|*{11}{c|}}
%  \hline
%   Parameter & \multicolumn{2}{|c|}{Aesthetic Images}  &  \multicolumn{2}{|c|}{Popular Vines} & \multicolumn{2}{|c|}{Unpopular Vines} \\ \hline
%   \_ & Mean&Median&Mean&Median&Mean&Median\\ \hline
%   \ Color Contrast & 51.05 & 30.22 & 29.88 &16.43 & 20.23  & 8.83  \\ \hline
%   \ Intensity Balance & 0.11 & 0.08 & 0.16 & 0.13 & 0.17 & 0.14 \\ \hline
%   \ Luo Simplicity & 0.009 & 0.005 & 0.013 & 0.012 & 0.015  & 0.014  \\ \hline
%   \ Sharp pixel proportion & 0.103 & 0.098 & 0.090 & 0.085 & 0.089  & 0.081  \\ \hline
%   \ Image Saturation & 0.943 & 0.974 & 0.672 & 0.678 & 0.615  & 0.646  \\ \hline
%   \ Avg. Brightness & 0.148 & 0.141 & 0.137 & 0.130 & 0.139  & 0.124  \\ \hline
%   \ Rule of Thirds & 0.879 & 0.899 & 0.883 & 0.883 & 0.878  &0.882  \\ \hline
%   \ ROI Proportion & 0.316 & 0.089 & 0.175 & 0.112 & 0.165  & 0.110  \\ \hline
%\end{tabular}}
%\label{aesthetic_table}
%\end{table}
%%%%%%%%%%%%%%%%%%%%%%%%%%%%%%%%%%%%%

\noindent\textbf{Audio Channel Comparison.} 
The audio channel is as important as the visual channel for long viral videos\footnote{\scriptsize http://thenextweb.com/socialmedia/2015/03/20/set-the-tone-the-importance-of-sound-in-viral-videos/}. In Fig. \ref{fig:comparisontable}, we see that in the audio space Vine and YouTube are far apart, with the highest K-L divergence value across all categories of features and all platform pairs (Note that the Flickr image dataset does not have an audio component). This is due to the fact that, while the audio tracks of Youtube Videos are very diverse, and therefore follow almost-uniform distributions across different feature ranges, all audio tracks in Vine videos tend to have very similar patterns. In Vine, audio is mainly a weak complement to the visual counterpart: Vine videos can be fully consumed and understood without the audio channel, and they are often played in the mute mode. As a matter of fact, Vine videos tend to have few rythmical changes (low \emph{Onset Rate}) and low \emph{Roughness}. Also, overall, due to their less curated audios, Vine videos tend to be louder than Youtube videos (high \emph{Energy}), as shown in Fig. \ref{fig:comparison_audio}.

\begin{figure}[!htb]
\centering
\includegraphics[width=\columnwidth]{plots/comparison/audio}
\caption{\textsl{ Distributions of the most diverging audio features across platforms.}}
\label{fig:comparison_audio}
\end{figure}

\begin{figure}[!htb]
\centering
\includegraphics[width=\columnwidth]{plots/comparison/objects}
\caption{\textsl{ Distributions of the most distant object occurrences across platforms, showing a preoccupation with fun and entertainment on Vine, `traditional' popular subjects such as kids, cars and violence on YouTube, and photographic scenery on Flickr.}}
\label{fig:comparison_objects}
\end{figure}

\noindent\textbf{Visual Semantics Comparison.} 
Finally, we examine platform differences in terms of the visual objects they focus on.  For each image in Flickr, and each sequence of images sampled from the YouTube and Vine videos, we use deep learning-based object detectors from ImageNet~\cite{krizhevsky2012imagenet}, and retain the labels of top-5 objects detected. We then aggregate such information at a platform level by computing the multinomial distribution of the detected objects for all Flickr images, Vine videos, and Youtube videos.  Such distributions reflect  the frequency of visual objects in typical popular videos of each platform (e.g. how often does a cat appear in a YouTube video?). Again, we then use symmetric KL divergence to compare object distributions across platforms. From Fig.~\ref{fig:comparisontable}, we see that, in the object space, Youtube and Flickr are equally distant from Vine. 

By looking at element-wise differences across distributions, we then rank objects according to how different their frequencies are for the 3 platforms, and report the top results in Fig.~\ref{fig:comparison_objects}. Vine can be clearly distinguished from the other 2 media due to the higher presence of objects related to celebrations, fun and entertainment (\emph{academic dress}, \emph{wig}, \emph{tv}, \emph{sunglasses}). Viral Youtube videos prefer popular subjects such as kids (\emph{diaper}), cars, and  violent scenes (\emph{punching}, \emph{neck brace}). Finally, Flickr pictures can be distinguished with the presence of visual concepts typical of photographic sceneries (\emph{lakeside}, \emph{seashore}). These results could be taken to indicate that the three platforms are used for different purposes, with micro-videos being more of an immediate medium, `capturing the moment' as it happens. This may help explain the nature of user engagement we observe in the rest of the paper.
%


