\begin{abstract}

With the impressive increase in scale of parallel processing, deep learning has started its journey to become an integral part of data science. 
The most interesting aspect of deep learning, is the fact that the network comes up with a higher dimensional representation
of the data, which is generally a very hard problem to achieve algorithmically. This property of deep neural networks has been widely used to 
show some impressive improvements in the fields of object recognition. Our paper explores the idea further and builds on top of some work in expression recognition, to explore the question :" What makes a video funny? ". More specifically, we explore the social perception of humour, in social mediums like Vine, which strive to deliver maximum impact in as less as 7 seconds. In this paper we further try to explore characteristics of top ranking videos and try to understand how different facial expressions contribute to the over all impact of humour in a vine video. We exploit the framework of supervised learning and build a hybrid model which analyses individual facial expressions along with their temporal variation, to come up with an insight into the abstract concept of humour. 

\end{abstract}