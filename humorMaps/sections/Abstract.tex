\section{Abstract}
Humour is one of the most fundamental human emotion which has a direct impact on how humans perceive
the context around them. Humour also plays a major part in what media shared over online social networks (OSNs) attains the status of being viral. With the advent of high impact social networks, which aim at delivering
condensed, compact media , new behaviours and genres of OSN media have evolved. Online social networks for delivery of social videos, like Vine, has changed the delivery and structure of humour completely. For a video on vine, the humorous impact has to be delivered in as less as 7 seconds, which implies the that the setup, context and the punchline has to be established in a condensed and high impact way. Our paper tries to understand this phenomenon of high impact of humour and perception of humour in high impact OSN media. We use state of the art data analysis and machine learning frameworks, and develop better methodologies in understanding the issue at hand. In the due course of the work, we also collect a month long data of popular vine videos for analysis of live data. 