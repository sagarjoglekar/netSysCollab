\begin{appendix}
\section{Appendix: Feature Table}



\begin{table}[hp]
  \centering
\resizebox{\linewidth}{!}{%
  \begin{tabular}{|c|r|c|p{17cm}|}
  \hline
	&  \textbf{Features} & \textbf{dim} & \multicolumn{1}{c|}{\textbf{Description}}\\
	  \hline
%    \multirow{15}{*}{\rotate{Content Features}}
    &\multicolumn{3}{c|}{\textbf{Visual Quality Features}} \\
	\cline{2-4}
		& RMS contrast & 1 & RMS contrast is calculated as standard deviation across all the pixels relative to mean intensity \\[4pt]
		&  Weber Contrast & 1 &  Weber contrast is  calculated as  $ F_\textit{weber} = \sum_{x = width}\sum_{y = height} \frac{I(x,y) - I_\textit{average}}{I_\textit{average}} $ \\[4pt]
		& Gray Contrast & 1 & Gray contrast is calculated in similar to RMS contrast in HSL colour space for the L value of pixels. \\ %[4pt]
		& Simplicity & 2 & Simplicity of composition of a photograph is a distinguishable factor that directly correlates with professionalism of the creator \cite{ke2006design}. We calculate Image simplicity by two methods: Yeh simplicity~\cite{yeh2010personalized} and Luo simplicity~\cite{luo2008photo}. \\ %[4pt]
		& Naturalness & 1 & How much does the image colors and objects match the real human perception?To compute image naturalness we convert the image into the HSV color space and  then identify pixels corresponding to natural objects like skin, grass, sky, water etc. This is done by considering pixels which an average brightness V \begin{math} \in \end{math} [20 , 80] and saturation S > 0.1. The final naturalness score is calculated by finding the weighted average of all the groups of pixels. \cite{predictingPintrest}. \\ %[4pt]
		& Colourfulness & 1 & A measure of colourfulness that describes the deviation from a pure gray image. It is calculated in RGB colour space as  
		$\sqrt{\sigma_\textit{rg}^2 + \sigma_\textit{yb}^2 } + 0.3\sqrt{\mu_\textit{rg}^2 + \mu_\textit{yb}^2}$ where $ \textit{rg} = \textit{R} - \textit{G}$ and $ yb = \frac{R + G}{2} $ and $\mu \text{ and }  \sigma $ represent mean and standard deviation respectively \\ %[4pt]
		& Hue Stats & 2 & Hue mean and variance which signifies the range of pure colours present in the image. It is directly derived from the HSL colour space \\ %[4pt]
		& LR balance &1 & Difference in intensity of pixels between two sections of an image is also a good measure of aesthetic quality. In non-ideal lighting conditions, images and videos tend to be over exposed in one part and correctly exposed in other. This is generally a sign of amateur creator. To capture this we compare the distribution of intensities of pixels in the left and right side of the image. The distance between the two distributions is measured using Chi-squared distance.\\
		& Rule of Thirds & 1 & This feature deals with compositional aspects of a photograph.This feature basically calculates if the object of interest is placed in one of the imaginary intersection of lines drawn at approximate one third of the horizontal and vertical positions. This is a well known aesthetic guideline for photographers. \\
		& ROI proportion & 1 & Measure of prominence given to salient objects. This measure detects the salient object in an image and then measures proportion of pixels its relative to the image \\
		& Image brightness & 3 & Features signify brightness of the image. Includes average brightness, saturation  and saturation variance\\
		& Image Sharpness & 1 & A measure of the clarity and level of detail of an image. Sharpness can be determined as a function of its Laplacian normalized by the local average luminance in the surroundings of each pixel, i.e. $\sum_{x, y} \frac{L(x, y)}{\mu_{xy}}$, with $L(x, y) = \frac{\partial^2I}{\partial x^2}+\frac{\partial^2I}{\partial y^2}$ where $\mu_xy$ denotes the average luminance around pixel (x, y).\\
		& Sharp Pixel Proportion  & 1 & Out of focus or blurry photographs are generally not considered aesthetically pleasing. In this feature we measure the proportion of sharp pixels compared to total pixels. We compute sharp pixels by converting the image in the frequency domain and then looking at the pixel corresponding to the regions of highest frequency \cite{yeh2010personalized}, using the OpenIMAJ \cite{Hare:2011:OIJ:2072298.2072421} tool.\\

	\cline{2-4}	
    &	\multicolumn{3}{c|}{\textbf{Higher Level Features}} \\
	\cline{2-4}
	 & Face Percentage & 1 & Percentage of frames in a video, which have been tested positive for at-least one face. Faces detected using Viola Jones Detector \cite{viola2004robust}\\
	 & Frame sentiment & 1 & Median frame sentiment of all the sampled frames from a micro video. The sentiment was calculated using the Multilingual Visual Sentiment 
	 Ontology detector \cite{jou2015visual} \\
	 & Past post count & 1 & Number of past posts user has uploaded prior to current one. This is a good measure of user's experience with the platform and activity.\\

	\cline{2-4}
    & \multicolumn{3}{c|}{\textbf{Audio Features}} \\
	\cline{2-4}
	& Zero Crossing rate & 1 & Zero crossing rate measures the rhythmic component an audio track \cite{laurier2009exploring}. It ends up detecting percussion instruments like Drums in the track\\
	& Loudness & 2 & This feature expresses overall perceived loudness as two components. Overall energy and average short time energy \cite{lartillot2007matlab}\\
	& Mode & 1 & This feature estimates the musical mode of the audio tract (major or minor). In western music theory, major modes give a perception of happiness and minor modes of sadness. \cite{laurier2009exploring} \\
	& Dissonance & 1 & Consonance and dissonance in an audio track has been shown to be relevant for emotional perception \cite{laurier2009exploring}. The values of dissonance are a calculate by measuring space between peaks in the frequency spectrum of the audio track. Consonant frequency peaks tend to be spaced evenly where as dissonant frequency peaks are not\\
	& Onset Rate & 1 & This measures the the Rhythmical perception. Onsets are peaks in the amplitude envelop of a track. Onset rate is measured by counting such events in a second. This typically gives a sense of speed to the track. \\
	\cline{2-4}
     &	\multicolumn{3}{c|}{\textbf{Social Features}} \\
	\cline{2-4}
	& Followers & 1 & Number of followers that the user posting a video has. This is the prime social feature available from the user meta-data. The number of followers directly represent the audience which are highly probably to engage with the video on upload.\\
	\hline
    \end{tabular}
    }
        \caption{Dimensionality and description of features used to describe Vine videos}
       \label{tab:Features_table}
%        \vspace{-5mm}
\end{table}

\end{appendix}
