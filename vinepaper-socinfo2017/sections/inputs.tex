To validate and understand the significance of the primacy property, it was important to understand relation of engagement of users with the primal states of a micro video. Following the methodology of training iterative and progressively selective classifier, we set up an experiment where the classifier would be trained on features extracted only from the initial 2 seconds of vine mico video instead of the entire video. We chose aesthetic and sentiment features extracted from frames sampled in the first two seconds and repeated the classifier training with the same set of hyper parameters for the classifier. It can be observed from the fig ~\ref{fig:Classifier_performance}, that the classifier trained of the primal sections of the videos, resulted in identical or in some cases better results, when compared to the classifier trained on the whole video. This validates the hypothesis that producers are deliberately or inadvertently creating contents that engage users in the first one third of the micro video. 
