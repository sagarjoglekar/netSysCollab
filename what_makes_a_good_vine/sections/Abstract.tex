\section{Abstract}
%Online content creators, both professional ones and amateur authors of user-generated content, need to answer an important question: How does one capture viewers' attention in this age of information overload? 
Users on social content-sharing websites are faced with a plethora of choice today, and make quick decisions about resharing or liking posts. How should content creators  capture viewers' attention in this age of information overload? Does quality of content matter? Or is there greater support for  ``rich gets richer'' theories, which suggest a self-perpetuating phenomenon where content from users with large numbers of followers stands a greater chance of becoming popular?  To the extent that quality matters, what aspect of the content is critical to ensuring popularity? We examine these questions using a snapshot of nearly all videos uploaded to vine over a 8 week period. We find that although social factors do affect popularity, content quality becomes critical at the top end of the popularity scale. Furthermore, using the temporal aspects of video, we verify that decisions are made quickly, and that first impressions matter more than deeper impressions. %FIX last sentence...


%In this article we examine the relevence of several aesthetic, sentimental and social features with popularity of microvideos. We look at influence of social and aesthetic features on how a vine video performs in the social world. We also look at the affective component of the videos to see if there are any peculiar sentiments related to the success of a vine video 
