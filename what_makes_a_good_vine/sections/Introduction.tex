\section{Introduction}

In the last few years, we have seen the introduction of a new form of user-generated video, where severe restrictions are placed on the duration of the content. High-profile examples include Vine, which allows users to create videos up to 6 seconds long; Instagram, which introduced videos up to 15 seconds long; and Snapchat, whose videos are officially limited to 10 seconds. Although most user-generated video platforms have placed some form of limit on the duration or size of videos (e.g., YouTube had a 10 minute limit, which has since been softened to a `default' limit of 15 mins\footnote{https://techcrunch.com/2010/12/09/youtube-time-limit-2/}), the extremely short duration time limits of Vine etc has led to the coining of a new term: \emph{micro videos}. 

On the one hand, these time restrictions offer a way for the platforms to manage the size of the videos and  thereby make the infrastructure management and content delivery more manageable. Thus, the time restrictions may be thought of as an artificial limit imposed on users. On the other hand, media commentators have argued that these restrictions could fundamentally change the way we communicate~\cite{bbc}. Similarly, David Pogue, writing in the Scientific American, conjectured that micro videos be seen as whole new form of expression that is more closely related to images than to videos~\cite{pogue13}:
\begin{quotation}
	A photograph is intended to capture a single moment, to present it for thoughtful examination. In the end, that's what a one- or six-second looping video does so well -- it's just that it expands the scope of the still image \ldots Maybe the micro video is best considered an improvement on a still picture, not a downgrade from video.
\end{quotation}

This  hypothesis raises the question of whether it is only \emph{conceptually} convenient and interesting to model  micro videos as enhanced images, or whether users also \emph{engage} with micro videos in ways similar to images. The answer has strong implications for how micro video platforms are designed and used. For instance, \ns{Miriam - pl check rest of this para and make it  sane from your perspective:} an image-like medium may benefit from image-like editing tools such as filters, cropping, fitting etc, whereas a video-like medium would benefit from non-linear or multi-track editing features. More importantly, many micro video platforms have started relaxing the time restrictions (e.g., Instagram videos can now last up to 60 seconds, and Vine recently introduced an option where users can attach a longer video up to 140 seconds long that can be reached in a single click from the 6 minute micro video). If in fact users see value in micro videos as images with embedded live action, then such attempts may alter  the very essence of the medium of expression. 

This paper takes a first look at how users engage with micro videos, by using Vine, one of the earliest and most popular micro video platforms, as a case study. %and also a platform with one of the shortest time duration restrictions (just over 6 seconds). 
We start by crawling POP12K, a dataset of about 12,000 videos which have been deemed by Vine to be popular, and therefore, by definition, have engaged a large number of users. We complement this by collecting ALL120K, a dataset of nearly all ($\approx$ 120,000)  videos that were uploaded to one of the 18 globally available channels on Vine. 

For each video in our dataset, we derive \ns{XXX} aesthetic and affect (sentiment)-related features (\ns{See Table XXX}) both in the video and the audio tracks, and collect statistics of the number of times each Vine was looped through, reposted or `revined', and liked by different users. Basing our study on these features, we systematically examine the nature of user engagement in Vine on three levels: First, we compare the features of micro videos in POP12K with corresponding features of popular content on image-based and video-based user generated content platforms (Flickr and YouTube respectively) and find evidence suggesting that Vine falls on a spectrum between images and videos. 

Next, we ask how these features vary over time in Vine videos, and discover a \emph{primacy of the first second} phenomenon: the best or most salient parts of the video, whether in the aesthetic space or affect space, are more prevalent in the initial seconds of the micro video, suggesting that the authors are consciously or unconsciously treating Vines similar to images -- in the initial part of the video, it is composed with aesthetics and affective quality in mind, resulting in a higher quality level; but quality declines 

Finally, we take the counts of loops, reposts and likes as metrics of collective user engagement of the \emph{consumers} of a video, and ask what factors affect these metrics. We develop a simple random forest classifier that is able to distinguish popular and unpopular items with high accuracy 

\hrule

\textbf{Below this is construction material for elsewhere}
As with other user-generated content, users may engage either as \emph{producers} of content or as \emph{consumers}. 

User engagement is both what users see and what they feel (cite). So we look at aesthetics as well as affect. Information comes in three channels: audio, video and sentiments. Vine engagement can be understood in terms of consumers and producers: what users produce and how they produce it. 

We examine this question in three 

the time restrictions shape user engagement with micro videos. To this end, we chose to examine Vine, one of the earliest and most popular micro video platforms, and also a platform with one of the shortest time duration restrictions (just over 6 seconds). We start by crawling POP12K, a dataset of about 12,000 videos which have been deemed by Vine to be popular, and therefore, have by definition, engaged a large number of users. We complement this by collecting ALL120K, a dataset of nearly all ($\approx$ 120,000)  videos that were uploaded to one of the 18 globally available channels on vine.

To understand which aspects impact engagement, we look at both the video and audio signals in the micro video. We consider \emph{aesthetics} -- how well constructed the video is, as well as affect -- the emotions the video can evoke. We ask if we can distinguish ``popular'' videos from the unpopular ones based on these different signals, and find that surprisingly, we can do a great

We first compare the popular micro videos in POP12K with popular user-generated images from Flickr and user-generated videos from YouTube, and find that micro videos appear to occupy a middle ground between images and videos. Then we ask whether and to what extent the

%In an age when 
%
%The Art of story telling could be attributed to be one of the most ancient arts. It could be in the form of neolithic paintings to the egyptian hieroglyphs or among the elaborate epics of Illiad and oddyssey to the elaborate power plays of the Game of thrones, humans have always strived to record or create elaborate plots and stories. The human need of transferring experiences to others in different forms of creative arts, has ever so created the world as interesting as we see it. 
%\par
%The art of story telling has spawned and transformed several industries, including the very important entertainment industry. With progress of technology, entertainment industry has gone through several rejuvination cycles. Starting with plays to television, each technological advancement has created a new form of stories to be presented. The latest of these cycles was powered by the internet with the help of services like Youtube, Netflix, Hulu and Amazon. 
%
%\par
%Over the past few years, the idea of microvideos has taken over the social world. It started with Vine, a company that was found in 2012, which allowed users to upload micro videos, not larger than 6 seconds. These videos are then consumed and shared, and are ranked based on how many times they are looped over. The videos have a spectrum of generes, starting from a quick home video about domestic cats and dogs, to elaborate short skits done by now acclaimed vine stars 
%
%\footnote{www.vanityfair.com/hollywood/2016/02/king-bach-rocketjump-youtube-vine-stars}
%\footnote{http://newmediarockstars.com/2015/04/youtubers-viners-attend-the-white-house-correspondents-dinner-gallery/}conceive of micro v