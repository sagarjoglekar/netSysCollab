\section{Discussion and conclusions}
Microvideos are no longer micro -- Instagram has extended from 15 -- 60 seconds; Vine has extended to 140 seconds, similar to parent company Twitter's restriction of 140 characters etc. As the restrictions become less stringent, it would be interesting to study whether the essence of microvideos will continue to be the same, because the modes of content production are the same (mobile phone-based apps), or whether microvideos start behaving more like videos.

We find support for \cite{pogue13}  who hypothesizes that microvideos bridge images and videos:
\begin{quotation}
	A photograph is intended to capture a single moment, to present it for thoughtful examination. In the end, that's what a one- or six-second looping video does so well-it's just that it expands the scope of the still image \ldots Maybe the micro video is best considered an improvement on a still picture, not a downgrade from video.
\end{quotation}