\section{ Related Work}
The work done in micro video analysis has been limited. Work by Miriam et.al \cite{redi20146} try to quantify and build on the notion of creativity. Work by \cite{reagan2016emotional} use textual sentiments to bring thousands of fiction novels to sentiment space and show that most novels follow 7 salient categories of stories. 
A paper by Nguyen et.al \cite{nguyen2016open} collected more than 200 thousand micro videos from vine. 
A work done by Fontanini et.al \cite{fontanini2016web} explore relevence of perceptual sentiments to popularity of a video, but the work done was on youtube viral videos, which have a much richer composition and structure. 
The problem of understanding what makes a visual media stick, has been a difficult one to solve. There are a few approaches to understand the aesthic and memorability aspects of an image \cite{Isola2011} \cite{datta2008algorithmic} \cite{goodSelfie}


\cite{Cha2009Flickr} --> Likes does not propagate quickly through Flickr social network (so quality is important)

\cite{Valafar2009} --> no correlation between age and popularity; most photos gain most likes in the first week. 

\cite{Yamasaki:2014} --> performs quite well, predicting popularity through text annotating  features.